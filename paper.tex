\documentclass{article}

%other packages
\usepackage{amsmath}
\usepackage{amssymb}
\usepackage{physics}

\usepackage[
	style=phys, articletitle=false, biblabel=brackets, chaptertitle=false, pageranges=false, url=true
]{biblatex}

\usepackage{graphicx}
\usepackage{todonotes}
\usepackage{siunitx}

\title{Title}

\addbibresource{./bibliography.bib}

\graphicspath{{./figures/}}

\newcommand{\vf}{v_{\mathrm{F}}}

\begin{document}

\maketitle

\section{Introduction \label{sec:intro}}
\begin{itemize}
	\item Motivate with~\cite{Tenberg2019} and~\cite{Kolkowitz2015}
	\item Existing work with Lindhard expression~\cite{QubitRelax}
	\item Variety of existing superconducting dielectric functions like in~\cite{AGD, llv9, Zimmermann1991, Mattis, Tinkham}.
	Advantage of Nam expression\cite{Nam1967} is that it's got the worked out dependence on momentum and impurities, both of which we need.
\end{itemize}
\section{Methods \label{sec:methods}}



	The relaxation rate for an electric qubit with dipole moment $\vec{d}$ a distance $r$ from a half space, described in~\cite{Henkel1999} and~\cite{QubitRelax}, can be written\todo{Add description here of how and why (e.g. Fermi's golden rule + FD theorem)}
	\begin{equation}
		\frac{1}{T_1} = \frac{d^2}{\hbar \epsilon_0} \frac{\omega^3}{c^3} \chi_{i}^{E}(r, \omega) \coth\frac{\omega}{2 T}.
	\end{equation}
	For a qubit pointing perpendicular to the surface of the half space,
	\begin{align}
		\chi_\perp^E(z, \omega) &= \Re \int_0^{+\infty} \dd{u} \frac{u^3}{v} e^{2 i z v} r_p(u). \label{eq:chi}
	\end{align}
	Here, $z$ is measured in units of the vacuum wavelength $\frac{c}{\omega}$, $v = \sqrt{1 - u^2}$ and we take the root $v = i \sqrt{u^2 - 1}$ for $u \geq 1$.

	The reflection coefficients are described by Ford and Weber~\cite{Ford1984}\todo{Add description of conditions and derivation methods, surface impedance are in quasistatic limit}:
	\begin{align}
		r_p(u) &= \frac{\pi v - \zeta_p(u)}{\pi v + \zeta_p(u)} \\
		r_s(u) &= \frac{\zeta_s(u) - \frac{\pi}{v}}{\zeta_s(u) + \frac{\pi}{v}} \\
		\zeta_p(u) &= 2i \int_0^\infty \dd{y} \frac{1}{\kappa^2} \left( \frac{y^2}{\epsilon(\frac{\omega}{c}\kappa, \omega) - \kappa^2} + \frac{u^2}{\epsilon_(\frac{\omega}{c}\kappa, \omega)} \right) \label{eq:zp} \\
		\zeta_s(u) &= 2i \int_0^\infty \dd{y} \frac{1}{\epsilon(\frac{\omega}{c}\kappa, \omega) - \kappa^2} \label{eq:zs} \\
		\kappa^2 &= u^2 + y^2
	\end{align}
	As noted in~\cite{QubitRelax}, these expressions remain valid even for nonlocal descriptions of the dielectric constant.

	Nam~\cite{Nam1967} describes the electrodynamics for superconductors applicable for clean and dirty materials.
	Assuming no magnetic impurities and weak coupling, his expressions reduce to
	\begin{equation}
		\epsilon(q, \omega) = 1 + \frac{3 \pi}{\omega^2} \frac{n e^2}{m} \left[\int_{\Delta - \omega}^{\Delta}\dd{\omega'} \tanh(\frac{\omega + \omega'}{2 T}) I_1 + \int_{\Delta}^{\infty} \dd{\omega'} \left( \tanh(\frac{\omega + \omega'}{2 T}) I_1  - \tanh(\frac{\omega'}{2 T})I_2 \right) \right],
	\end{equation}
	with
	\begin{align}
		I_1 &= F(q, \Re[\sqrt{(\omega + \omega')^2 - \Delta^2} - \sqrt{\omega'^2 - \Delta^2}]) (g + 1) \nonumber\\
		&\quad + F(q, \Re[-\sqrt{(\omega + \omega')^2 - \Delta^2} - \sqrt{\omega'^2 - \Delta^2}]) (g - 1) \\
		I_2 &= F(q, \Re[\sqrt{(\omega + \omega')^2 - \Delta^2} - \sqrt{\omega'^2 - \Delta^2}]) (g + 1) \nonumber\\
		&\quad + F(q, \Re[\sqrt{(\omega +  \omega')^2 - \Delta^2} + \sqrt{\omega'^2 - \Delta^2}]) (g - 1) \\
		F(q, E) &= \frac{1}{q \vf} \left[2 S(E) + (1 - S(E)^2)\ln(\frac{S(E) + 1}{S(E) - 1})\right]  \\
		S(q, E) &= \frac{1}{q \vf} \left( E - i \left(\Im[\sqrt{(\omega + \omega')^2 - \Delta^2} + \sqrt{\omega'^2 - \Delta^2}] + \frac{2}{\tau} \right) \right) \\
		g &= \frac{\omega' \left(\omega + \omega'\right) + \Delta^2}{\sqrt{\omega'^2 - \Delta^2}\sqrt{(\omega + \omega')^2 - \Delta^2}}.
	\end{align}

	As seen in figure\todo{Insert 3D plot here}, the temperature dependence for the superconducting state is richer than for the normal state described in~\cite{QubitRelax}.
	When $\omega$ and $T$ are both much smaller than $T_c$, the figure reveals the expected reduction in noise.
	As $T \rightarrow T_c$, a smaller $\omega$ is required to facilitate Johnson noise.
	In figure\todo{Insert chi vs T plot, for Lindhard and Nam}, the more sensitive temperature dependence of the superconductor is visible against a normal state calculation done with the Lindhard function, as in~\cite{QubitRelax}.
	For $T \rightarrow T_c$, the superconducting state noise approaches that of the normal state, as expected.

\section{Numerical Techniques \label{sec:technical}}

	Nam's expressions are no longer valid for $k \geq k_\mathrm{F}$, and as $k \rightarrow \infty$ exhibit an unphysical $\flatfrac{1}{k}$ dependence.
	In order to use~\eqref{eq:zp} and~\eqref{eq:zs}
\begin{itemize}
	\item Limitations on superconducting dielectric function.
	In every expression that's worked out for arbitrary momentum, the large-momentum dependence goes as $\frac{1}{k}$, which is clearly unphysical.
	Issue exists in~\cite{Nam1967, AGD, Tinkham}, etc.
	\item Include description of cutoff process.
	\item Include description of the interpolation process.
	\item Is the numerics of picking the relevant unit scales / discussion of convergence interesting?
	It might be too obvious that the convergence is expected because these are real quantities.
	\item Any other potentially interesting approximations?
	Some integrals have regions entirely discarded because they are too small to matter, which is why the expressions in the Mathematica package are much simpler than what's in Nam.
\end{itemize}

\section{Experiments \label{sec:experiments}}
\begin{itemize}
	\item Show that expression is of right order of magnitude to describe~\cite{Tenberg2019}.
	\item Look at temperature at which~\cite{Tenberg2019} should show SC effects (not far off!)
	\item Compare to~\cite{Kolkowitz2015}?
	Much further from relevant temperature range.
	\item Metal choices?
	Obviously large $T_c$ is most important to get noise reduction benefits.
	What else is most relevant for usable metals?
\end{itemize}
\section{Conclusions \label{sec:conclusions}}

\printbibliography

\end{document}

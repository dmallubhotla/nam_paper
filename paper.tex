\documentclass{article}

%other packages
\usepackage{amsmath}
\usepackage{amssymb}
\usepackage{physics}

\usepackage[
	style=phys, articletitle=false, biblabel=brackets, chaptertitle=false, pageranges=false, url=true
]{biblatex}

\usepackage{graphicx}
\usepackage{todonotes}
\usepackage{siunitx}

\title{Title}

\addbibresource{./bibliography.bib}

\graphicspath{{./figures/}}

\newcommand{\vf}{v_{\mathrm{F}}}

\begin{document}

\maketitle

\section{Introduction \label{sec:intro}}
\begin{itemize}
	\item Motivate with~\cite{Tenberg2019} and~\cite{Kolkowitz2015}
	\item Existing work with Lindhard expression~\cite{QubitRelax}
	\item Variety of existing superconducting dielectric functions like in~\cite{AGD, llv9, Zimmermann1991, Mattis, Tinkham}.
	Advantage of Nam expression\cite{Nam1967} is that it's got the worked out dependence on momentum and impurities, both of which we need.
\end{itemize}
\section{Methods \label{sec:methods}}



	The relaxation rate for an electric qubit with dipole moment $\vec{d}$ a distance $r$ from a half space, described in~\cite{Henkel1999} and~\cite{QubitRelax}, can be written\todo{Add description here of how and why (e.g. Fermi's golden rule + FD theorem)}
	\begin{equation}
		\frac{1}{T_1} = \frac{d^2}{\hbar \epsilon_0} \frac{\omega^3}{c^3} \chi_{i}^{E}(r, \omega) \coth\frac{\omega}{2 T}.
	\end{equation}
	For a qubit pointing perpendicular to the surface of the half space,
	\begin{align}
		\chi_\perp^E(z, \omega) &= \Re \int_0^{+\infty} \dd{u} \frac{u^3}{v} e^{2 i z v} r_p(u). \label{eq:chi}
	\end{align}
	Here, $z$ is measured in units of the vacuum wavelength $\frac{c}{\omega}$, $v = \sqrt{1 - u^2}$ and we take the root $v = i \sqrt{u^2 - 1}$ for $u \geq 1$.

	The reflection coefficients are described by Ford and Weber~\cite{Ford1984}\todo{Add description of conditions and derivation methods, surface impedance are in quasistatic limit}:
	\begin{align}
		r_p(u) &= \frac{\pi v - \zeta_p(u)}{\pi v + \zeta_p(u)} \\
		r_s(u) &= \frac{\zeta_s(u) - \frac{\pi}{v}}{\zeta_s(u) + \frac{\pi}{v}} \\
		\zeta_p(u) &= 2i \int_0^\infty \dd{y} \frac{1}{\kappa^2} \left( \frac{y^2}{\epsilon(\frac{\omega}{c}\kappa, \omega) - \kappa^2} + \frac{u^2}{\epsilon_(\frac{\omega}{c}\kappa, \omega)} \right) \label{eq:zp} \\
		\zeta_s(u) &= 2i \int_0^\infty \dd{y} \frac{1}{\epsilon(\frac{\omega}{c}\kappa, \omega) - \kappa^2} \label{eq:zs} \\
		\kappa^2 &= u^2 + y^2
	\end{align}
	As noted in~\cite{QubitRelax}, these expressions remain valid even for nonlocal descriptions of the dielectric constant.

	Nam~\cite{Nam1967} describes the electrodynamics for superconductors applicable for clean and dirty materials.\todo{Describe his derivations here, based on Green's function methods \& can generalise to strong-coupling and magnetic impurities}
	Assuming no magnetic impurities and weak coupling, his expressions reduce to
	\begin{equation}
		\epsilon(q, \omega) = 1 + \frac{3 \pi}{\omega^2} \frac{n e^2}{m} \left[\int_{\Delta - \omega}^{\Delta}\dd{\omega'} \tanh(\frac{\omega + \omega'}{2 T}) I_1 + \int_{\Delta}^{\infty} \dd{\omega'} \left( \tanh(\frac{\omega + \omega'}{2 T}) I_1  - \tanh(\frac{\omega'}{2 T})I_2 \right) \right], \label{eq:eps}
	\end{equation}
	with
	\begin{align}
		I_1 &= F(q, \Re[\sqrt{(\omega + \omega')^2 - \Delta^2} - \sqrt{\omega'^2 - \Delta^2}]) (g + 1) \nonumber\\
		&\quad + F(q, \Re[-\sqrt{(\omega + \omega')^2 - \Delta^2} - \sqrt{\omega'^2 - \Delta^2}]) (g - 1) \\
		I_2 &= F(q, \Re[\sqrt{(\omega + \omega')^2 - \Delta^2} - \sqrt{\omega'^2 - \Delta^2}]) (g + 1) \nonumber\\
		&\quad + F(q, \Re[\sqrt{(\omega +  \omega')^2 - \Delta^2} + \sqrt{\omega'^2 - \Delta^2}]) (g - 1) \\
		F(q, E) &= \frac{1}{q \vf} \left[2 S(E) + (1 - S(E)^2)\ln(\frac{S(E) + 1}{S(E) - 1})\right]  \\
		S(q, E) &= \frac{1}{q \vf} \left( E - i \left(\Im[\sqrt{(\omega + \omega')^2 - \Delta^2} + \sqrt{\omega'^2 - \Delta^2}] + \frac{2}{\tau} \right) \right) \\
		g &= \frac{\omega' \left(\omega + \omega'\right) + \Delta^2}{\sqrt{\omega'^2 - \Delta^2}\sqrt{(\omega + \omega')^2 - \Delta^2}}.
	\end{align}

	As seen in figure\todo{Insert 3D plot here}, the temperature dependence for the superconducting state is richer than for the normal state described in~\cite{QubitRelax}.
	When $\omega$ and $T$ are both much smaller than $T_c$, the figure reveals the expected reduction in noise.
	As $T \rightarrow T_c$, a smaller $\omega$ is required to facilitate Johnson noise.
	In figure\todo{Insert chi vs T plot, for Lindhard and Nam}, the more sensitive temperature dependence of the superconductor is visible against a normal state calculation done with the Lindhard function, as in~\cite{QubitRelax}.
	For $T \rightarrow T_c$, the superconducting state noise approaches that of the normal state, as expected.

\section{Numerical Techniques \label{sec:technical}}

	Nam's expressions are no longer valid for $q \geq q_\mathrm{F}$, and as $q \rightarrow \infty$ exhibit an unphysical $\flatfrac{1}{q}$ dependence, a feature shared by similar expressions in \cite{AGD}.
	In order to use~\eqref{eq:zp} and~\eqref{eq:zs}, this must be corrected to prevent divergences.
	For sufficiently large momenta, the response function should approach the normal state function\todo{add good ref of this}.
	Moreover, $\Im \epsilon$ should go to $0$ when $q \gtrapprox 2 k_{\mathrm{F}}$, as otherwise there will be no available states within the Fermi surface for energy transfer (for more on this point, see the discussion in~\cite{FetterWalecka}).

	In order to account for the first point, for the normal state we can use the Lindhard dielectric function, which has the correct nonlocal behaviour to describe the low $z$ noise\cite{QubitRelax}.
	As described from~\cite{SolyomV3}, this is
	\begin{equation}
		\epsilon_{\mathrm{Lindhard}}(\vec{q}, \omega) = 1 + \frac{q_{TF}^2}{q^2}\frac{\displaystyle 1 + \frac{\omega + \flatfrac{i}{\tau}}{2 \vf q} \ln(\frac{\omega - \vf q + \flatfrac{i}{\tau}}{\omega + \vf q + \flatfrac{i}{\tau}})}{\displaystyle 1 + \frac{\flatfrac{i}{\tau}}{2 \vf q} \ln(\frac{\omega - \vf q + \flatfrac{i}{\tau}}{\omega + \vf q + \flatfrac{i}{\tau}})}. \label{eq:lindhardsolyom}
	\end{equation}

	The effect of the inaccurate large momentum values in $\epsilon$ is an overestimation of the dissipative part $\Im r_p$ of the reflection coefficient.
	To correct for this, we can use~\eqref{eq:lindhardsolyom} and~\eqref{eq:eps} to find $r_{p\mathrm{, Lindhard}}$ and $r_{p\mathrm{, Nam}}$, then choose whichever value has a smaller imaginary part.
	Effectively, this defines an $r_{p\mathrm{, effective}}$.
	For $q$ below some cutoff $q_{uc}$ on the order of $q_{\mathrm{F}}$, this picks the Nam reflection coefficient and reflects the diminished noise in the superconducting state.\todo{Does this need to be justified with a graph of the two functions?}

	The integral in~\eqref{eq:chi} picks out values around $u = \frac{c}{\omega} q \sim \frac{1}{z}$.
	As long as $\frac{1}{z} \ll \frac{\omega}{c} q_{uc}$, the value for the noise will reflect the effects of the superconducting expressions, without picking up the inaccuracies of the transition region.


\section{Experiments \label{sec:experiments}}
\begin{itemize}
	\item Show that expression is of right order of magnitude to describe~\cite{Tenberg2019}.
	\item Look at temperature at which~\cite{Tenberg2019} should show SC effects (not far off!)
	\item Compare to~\cite{Kolkowitz2015}?
	Much further from relevant temperature range.
	\item Metal choices?
	Obviously large $T_c$ is most important to get noise reduction benefits.
	What else is most relevant for usable metals?
\end{itemize}
\section{Conclusions \label{sec:conclusions}}

\printbibliography

\end{document}
